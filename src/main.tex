\documentclass[dvipdfmx,b4j]{jsarticle}
\usepackage{main}
%theorem->定理
%prop->命題
%cor->系
%definition->定義
%lemma->補題
%example->例
\begin{document}
\title{現代数学入門}
\author{Hzo,Alps}

\maketitle
\tableofcontents


\section{}
\subsection{集合}
\subsubsection{}
\begin{definition}{集合}{}
\textbf{集合}(set)とは,数学的対象の集まりのことを指す.
集合Xに属する対象を$X$の\textbf{元}(element)あるいは\textbf{要素}といい,
$x$が$X$の元であることを$x\in X$と表す.
\end{definition}
\noindent $\bf{Note}$:集合では元の並び順や,同一の元がいくつあるかなどは考えない
\begin{itemize}
\item 内包的記法,外延的記法
\item 空集合
\end{itemize}

\subsubsection{}
\begin{definition}{部分集合}{}
$X,Y$を集合とする.$x\in X\Longrightarrow x\in Y$が成り立つとき,
$X$は$Y$の\textbf{部分集合}(subset)である,
$X$は$Y$に含まれるなどといい,$X\subset Y$と書く.
\end{definition}
\begin{definition}{集合の相当}{}
$A\subset B $かつ$A\supset B$であるとき,集合$A$と集合$B$は等しいとし,
$A = B $と書く.
\end{definition}
\begin{itemize}
    \item 真部分集合
    \item 補集合
    \item de Morganの法則
\end{itemize}
%
%集合では,元の並び順や,同一の元がいくつあるかなどは考えない
%上のコメントに関する例題
%%%%
%%%%練習問題
\begin{oframed}
\noindent\textbf{例題}
\begin{enumerate}
    \item 全ての集合は空集合を部分集合として含むことを示せ.
    \item $X\subset Y$かつ,$Y\subset Z$ならば$X\subset Z$であることを示せ.%変える可能性あり.
\end{enumerate}
\end{oframed}
%%%%
%%%%
\subsubsection{}
\begin{definition}{和集合・共通部分}{}
$X$の部分集合族$\{Y_\lambda\mid\lambda\in\Lambda\}$の\textbf{和集合}$\cup_{\lambda\in\Lambda}Y\lambda$は,
どれかの$Y_\lambda$の元になっている元全体の集合で,
\textbf{共通部分}$\cap_{\lambda\in\Lambda}Y_\lambda$はどの$Y_\lambda$の元にもなっている元全体の集合である.つまり,
\begin{align*}
&\bigcup_{\lambda\in\Lambda}Y_\lambda\quad=\{x\in X\mid \text{ある}\lambda\text{に対しても}x\in Y_\lambda\}\\
&\bigcap_{\lambda\in\Lambda}Y_\lambda\quad=\{x\in X\mid \text{どの}\lambda\text{に対しても}x\in Y_\lambda\}\\
\end{align*}

\end{definition}
\begin{itemize}
\item 集合差
\item 積集合
\item 冪集合
\end{itemize}
\subsection{写像}
\end{document}
