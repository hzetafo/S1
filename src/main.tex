\documentclass[dvipdfmx,b4j]{jsarticle}
\usepackage{main}
%theorem->定理
%prop->命題
%cor->系
%definition->定義
%lemma->補題
%example->例

%tcbline->tcolorboxのhrulefill
%英訳を書く際は\noindent\rule{\linewidth}{0.4pt}
\begin{document}
\title{現代数学}
\author{Hzo,Alps}
%%%%記号法を入れる
%%%%
\maketitle
\begin{framed}
\textbf{注意}
\begin{enumerate}
    \item 節末に記される概念は書中で詳述されていないため,外部資料で補完することが望ましい.
    \item 本書の読解と並行して\TeX の習得を進めることが望ましい.
    \item 演習問題をおろそかにせぬこと.
\end{enumerate}
\end{framed}
\tableofcontents

\section{}
\subsection{集合}
\subsubsection{}
\begin{definition}{集合}{}
\textbf{集合}(set)とは,数学的対象の集まりのことを指す.集合Xに属する対象を$X$の\textbf{元}(element)あるいは\textbf{要素}といい,$x$が$X$の元であることを$x\in X$と表す.
\end{definition}
\noindent $\bf{Note1}$:集合では元の並び順や,同一の元がいくつあるかなどは考えない
\begin{itemize}
\item 内包的記法,外延的記法
%\item 集合族
\item 空集合
\end{itemize}

\subsubsection{}
\begin{definition}{部分集合}{}
$X,Y$を集合とする.$x\in X\Longrightarrow x\in Y$が成り立つとき,$X$は$Y$の\textbf{部分集合}(subset)である,$X$は$Y$に含まれるなどといい,$X\subset Y$と書く.
\end{definition}
\begin{definition}{集合の相当}{}
$A\subset B $かつ$A\supset B$であるとき,集合$A$と集合$B$は等しいとし,$A = B $と書く.
\end{definition}
\begin{itemize}
    \item 真部分集合
    \item 補集合
    \item de Morganの法則
\end{itemize}

%%%%
%%%%練習問題
%\begin{oframed}
%\noindent\textbf{例題}
%\begin{enumerate}
%    \item 全ての集合は空集合を部分集合として含むことを示せ.
%    \item $X\subset Y$かつ,$Y\subset Z$ならば$X\subset Z$であることを示せ.%変える可能性あり.
%\end{enumerate}
%\end{oframed}
%%%%
%%%%
\subsubsection{}
\begin{definition}{和集合・共通部分}{}
$X$の部分集合族$\{Y_\lambda\mid\lambda\in\Lambda\}$の\textbf{和集合}$\cup_{\lambda\in\Lambda}Y\lambda$は,
どれかの$Y_\lambda$の元になっている元全体の集合で,
\textbf{共通部分}$\cap_{\lambda\in\Lambda}Y_\lambda$はどの$Y_\lambda$の元にもなっている元全体の集合である.つまり,
\begin{align*}
&\bigcup_{\lambda\in\Lambda}Y_\lambda\quad=\{x\in X\mid \text{ある}\lambda\text{に対しても}x\in Y_\lambda\}\\
&\bigcap_{\lambda\in\Lambda}Y_\lambda\quad=\{x\in X\mid \text{どの}\lambda\text{に対しても}x\in Y_\lambda\}\\
\end{align*}
\tcbline
\begin{itemize}
    \item \textbf{和集合}:union
    \item \textbf{共通部分}:intersection
\end{itemize}
\end{definition}
\begin{itemize}
\item 集合差
\item 直積集合
\item 冪集合
\item 対象差
\end{itemize}
%%%%
%%%%直積集合/直積集合の普遍性
%%%%直和集合/直和の普遍性
%%%%
%%%%
\subsection{写像}
\subsubsection{}
\begin{definition}{写像}{}
集合$X$から集合$Y$への\textbf{写像}$f$とは,$X$がどの元$x$に対しても$Y$の元$y$が1通りに定まってることを言い,$f\colon X\to Y$とか$X\stackrel{f}{\to}Y$とか書く.
このとき,$y$を$f$による\textbf{像}と言い,$y=f(x)$と書く.また$Y$が数の集合であるときは,$f$を$X$上の\textbf{関数}と言うことがある.
%英訳
\tcbline
\begin{itemize}
    \item \textbf{写像}:Mapping,Map
    \item \textbf{像}:Image
    \item \textbf{関数}:function
\end{itemize}
%
\end{definition}
\noindent $\bf{Note1}$:オイラーによる関数の定義は,変数や定数から構成される数式を基にしていた.一方,現代の定義では,関数は独立変数と従属変数の任意の対応関係として捉えられ,必ずしも数式で表現される必要はない.

\noindent $\bf{Note2}$:$y$ に対して $x$ が一意に定まるとは限らないことに注意する.
\begin{itemize}
  \item 単射,全射,全単射
  \item 写像の相当
  \item 制限写像
  \item 写像の合成
  \item 包含写像,恒等写像
\end{itemize}
%%%
\clearpage
\centerline{\huge{\textbf{第一章のまとめ}}}
\begin{shaded}
\noindent\textbf{定義した概念}
\begin{multicols}{2}
\begin{itemize}
  \item 集合,元
  \item 内包的記法,外延的記法
%  \item 集合族 <-添字集合
  \item 空集合 
  \item 部分集合
  \item 真部分集合
  \item 補集合
  \item de Morganの法則
  \item 和集合,共通部分
  \item 集合差
  \item 直積集合
  \item 冪集合
  \item 対象差
  \item 写像,像,関数
  \item 単射,全射,全単射
  \item 写像の相当
  \item 制限写像
  \item 写像の合成
  \item 包含写像,恒等写像
\end{itemize}
\end{multicols}
%選択公理
\end{shaded}
\clearpage

\centering{\textbf{演習問題}}

\end{document}
