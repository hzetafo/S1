\documentclass[dvipdfmx,b4j]{jsarticle}
\usepackage{main}
%theorem->定理
%prop->命題
%cor->系
%definition->定義
%lemma->補題
%example->例
\begin{document}
\tableofcontents

\section{}
\subsection{集合}

\subsubsection{}
\begin{definition}{集合}{}
\textbf{集合}(set)とは,数学的対象の集まりのことを指す.
集合Xに属する対象を$X$の\textbf{元}(element)あるいは\textbf{要素}といい,
$x$が$X$の元であることを$x\in X$と表す.
\end{definition}
%内包的記法,外延的記法
%集合には二通りの表し方があり,
%
%
\subsubsection{}
\begin{definition}{部分集合}{}
$X,Y$を集合とする.$x\in X\Longrightarrow x\in Y$が成り立つとき,
$X$は$Y$の\textbf{部分集合}(subset)である,
$X$は$Y$に含まれるなどといい,$X\subset Y$と書く.
\end{definition}

\subsection{写像}
\end{document}
