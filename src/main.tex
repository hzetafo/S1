\documentclass[dvipdfmx,b4j]{jsarticle}
\usepackage{main}
%theorem->定理
%prop->命題
%cor->系
%definition->定義
%lemma->補題
%example->例
\begin{document}

%%
\tableofcontents
\section{}

\subsection{集合}

\subsubsection{}
\begin{definition}{集合}{}
\textbf{集合}(set)とは,数学的対象の集まりのことを指す.
集合Xに属する対象を$X$の\textbf{元}(element)あるいは\textbf{要素}といい,
$x$が$X$の元であることを$x\in X$と表す.
\end{definition}
%内包的記法,外延的記法
\subsubsection{}
\begin{definition}{部分集合}{}
$X,Y$を集合とする.$x\in X\Longrightarrow x\in Y$が成り立つとき,
$X$は$Y$の\textbf{部分集合}(subset)である,
$X$は$Y$に含まれるなどといい,$X\subset Y$と書く.
\end{definition}
\subsubsection{補集合}
\begin{definition}{補集合}{}
部分集合$Y$の$X$における補集合を
$$Y^C=\{x\in X|x\not\in Y\}$$
\end{definition}
\subsubsection{}
\begin{definition}{空集合}{}
元を一つも含まない集合を空集合とよび$\emptyset$で表す.
\end{definition}
\begin{definition}{互いに素}{}
$A,B$を集合$X$の部分集合とする.$A,B$が互いに素(dis-joint)であるとは,$A\cap B=\emptyset $ 
\end{definition}
\subsubsection{}
\begin{definition}{}{}
$X,Y$を集合とする.このとき,集合$X\cap Y$をおよび$X\cup Y$を
$$$$%%%%
として定める.
\end{definition}
\subsection{写像}
\end{document}
